\documentclass[prd, nofootinbib, floatfix, 11pt,tightenlines,times]{article}

\usepackage[paperwidth=8.5in,paperheight=11in,centering,margin=1in]{geometry}
\usepackage{amsmath}
\usepackage{amsbsy}
\usepackage{natbib}
\usepackage{rotating}
\input epsf
\usepackage{amsmath}
\usepackage{wasysym}
\usepackage{subfigure}
\usepackage{graphicx}
\usepackage{epsfig}
\usepackage{color}
%\usepackage{ulem}
%\usepackage{epstopdf}

%\renewcommand{\baselinestretch}{0.98}
\usepackage{epsfig}
\usepackage{titletoc}

% HOW TO SET UP AN 8.5 x 11:
% http://www.pages.drexel.edu/~pyo22/students/latexRelated/latexTutorial.html
\topmargin -1.5cm        % read Lamport p.163
\oddsidemargin -0.04cm   % read Lamport p.163
\evensidemargin -0.04cm  % same as oddsidemargin but for left-hand pages
\textwidth 16.59cm
\textheight 21.94cm 
\parskip 7.2pt           % sets spacing between paragraphs
\parindent 0pt		     % sets leading space for paragraphs

\author{Andrew Becker, Simon Krughoff, Andrew Connolly, Russell Owen}
\title{Summary of Late Winter2013 Production: ip\_diffim}
\date{\today}

\begin{document}

\maketitle

We summarize

\clearpage
\tableofcontents
\clearpage

\section{KernelCandidate Statistics}

\subsection{Mean Squared Error}

In order to investigate the bias--variance trade--off in the overall
fit, we calculate the MSE of each KernelCandidate's difference image
derived from the evaluation of the spatial model at that location.  We
define the bias as $\left| data - model \right|$, the variance as
$\left| (data - model)^2 \right|$, and the MSE as {\tt bias$^2$ +
  variance}.  In this context, the bias is the mean of the difference
image, and the variance is the mean square of the difference image.
In both cases, we normalize by the square root of the variance so that
pixels are in units of sigma.

\subsection{Reduced $\chi^2$}

\section{Dependencies of False Positive Rate}

After finding the configurations that yielded the minimal numbers of
false positives, we then perturbed the solutions to examine how the
numbers of false positives scaled with different effects.  This
included, in order of importance for this production: the difference
image detection threshold; the number of KernelCandidates going into
the spatial model; and astrometric registration errors.

\subsection{Detection Threshold}

\subsection{Number of KernelCandidates}

\subsection{Registration Errors}

We implemented two simple perturbations of the inputs to the
image--to--image RegisterTask: we first added a random offset to each
object's (x,y)-coordinates, with an amplitude that was specified in
the Config and multiplied by a random number pulled from a normal
distribution; and we added a DC offset to the coordinates at an
amplitude specified in the Config.  These offsets, added to the Source
coordinates, will affect misalignments of the objects in the
registered images, as the registration is done assuming the specified
positions are correct.  In this way we are able to investigate how
random uncertainties and bulk astrometric offsets impact the false
positive rate.  We explicitly do {\it not} investigate spatial
variation in these offsets, using for example a pincushion distortion.
We anticipate that this latter effect will be most important for
spatial interpolation and extrapolation of the matching kernel,
yielding a dipole residual field associated with the distortion.
However, the ability of the software to model out these spatial
distortions will certainly fail if the local kernel solutions (from
which the spatial model is derived) are unable to compensate for
misalignments.

\subsubsection{Coordinate RMS}

We perturbed the coordinates of each template Source that was input to
RegisterTask with amplitudes of (0.025, 0.05, 0.075, 0.1, 0.125, 0.15,
0.175, 0.2, 0.3, 0.4) pixels, and to randomize the offsets multiplied
each offset by a number pulled from a normal (0,1) distribution.  The
output RMS reported by RegisterTask was noted to track these offsets.

The false positive rate was not seen to vary significantly.

\subsubsection{Coordinate Offsets}

We offset the coordinates of each template Source that was input to
RegisterTask with amplitudes of (0.025, 0.05, 0.075, 0.1, 0.125, 0.15,
0.175, 0.2, 0.3, 0.4, 1.0, 1.5, 2.0, 2.5, 3.0, 3.5, 4.0) pixels.
Post--registration, this will offset the positions of sources in the
two images by the desired amount.

\end{document}
